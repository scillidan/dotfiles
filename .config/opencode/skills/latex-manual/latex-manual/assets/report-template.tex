% Report Template
% For longer documents with chapters (thesis, technical reports)

\documentclass[12pt,a4paper]{report}

% ===== PACKAGES =====
\usepackage[utf8]{inputenc}
\usepackage[T1]{fontenc}
\usepackage[margin=1in]{geometry}
\usepackage{graphicx}
\usepackage{amsmath,amssymb,amsthm}
\usepackage{hyperref}
\usepackage{cite}
\usepackage{booktabs}
\usepackage{setspace}
\usepackage{fancyhdr}

% ===== PAGE STYLE =====
\pagestyle{fancy}
\fancyhf{}
\fancyhead[L]{\leftmark}
\fancyhead[R]{\thepage}
\renewcommand{\headrulewidth}{0.4pt}

% Line spacing
\onehalfspacing

% ===== HYPERREF CONFIGURATION =====
\hypersetup{
    colorlinks=true,
    linkcolor=blue,
    citecolor=blue,
    urlcolor=cyan,
    pdftitle={Report Title},
    pdfauthor={Author Name}
}

% ===== TITLE PAGE INFORMATION =====
\title{
    {\LARGE\textbf{Report Title}}\\[1cm]
    {\large Subtitle if needed}
}
\author{
    Author Name\\
    \texttt{email@example.com}\\[0.5cm]
    Institution Name\\
    Department Name
}
\date{\today}

% ===== DOCUMENT START =====
\begin{document}

% ===== FRONT MATTER =====
\pagenumbering{roman}

\maketitle

\begin{abstract}
This is the abstract. Provide a concise summary of the entire report, including objectives, methods, key findings, and conclusions. Typically 250-500 words for a report.
\end{abstract}

\tableofcontents
\listoffigures
\listoftables

% ===== MAIN MATTER =====
\clearpage
\pagenumbering{arabic}

\chapter{Introduction}
\label{ch:introduction}

This is the introduction chapter. Reports use chapters as the top-level division.

\section{Background}
\label{sec:background}

Provide context and background information.

\section{Objectives}
\label{sec:objectives}

State the objectives of this report clearly:
\begin{enumerate}
    \item First objective
    \item Second objective
    \item Third objective
\end{enumerate}

\section{Scope}
\label{sec:scope}

Define what is and isn't covered in this report.

\chapter{Literature Review}
\label{ch:literature}

Review relevant previous work and research.

\section{Topic Area 1}

Discuss first major topic area.

\section{Topic Area 2}

Discuss second major topic area.

\chapter{Methodology}
\label{ch:methodology}

Describe your methods, approach, or experimental design.

\section{Research Design}

Explain your overall approach.

\section{Data Collection}

Describe how data was collected.

\section{Analysis Methods}

Explain analytical techniques used.

\chapter{Results}
\label{ch:results}

Present your findings systematically.

\section{Key Finding 1}

Present first major finding with supporting evidence.

\begin{figure}[h]
    \centering
    % \includegraphics[width=0.8\textwidth]{figures/result1.pdf}
    \caption{Example results figure}
    \label{fig:result1}
\end{figure}

\section{Key Finding 2}

Present second major finding.

\chapter{Discussion}
\label{ch:discussion}

Interpret your results and discuss their implications.

\section{Interpretation of Results}

What do the results mean?

\section{Limitations}

Acknowledge limitations of your work.

\section{Implications}

Discuss broader implications.

\chapter{Conclusion}
\label{ch:conclusion}

\section{Summary of Findings}

Recap the main findings.

\section{Recommendations}

Provide actionable recommendations.

\section{Future Work}

Suggest directions for future research or development.

% ===== BIBLIOGRAPHY =====
\bibliographystyle{plain}
% \bibliography{references}

% Manual bibliography if not using BibTeX
\begin{thebibliography}{99}
    \bibitem{example}
        Author, A. (2024). \textit{Title of Work}. Publisher.
\end{thebibliography}

% ===== APPENDICES =====
\appendix

\chapter{Additional Data}
\label{app:data}

Supplementary material, detailed data, or extended proofs.

\chapter{Code Listings}
\label{app:code}

Source code or detailed algorithms.

\end{document}
