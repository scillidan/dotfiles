% Article Template
% Basic structure for research papers, articles, and essays

\documentclass[12pt,a4paper]{article}

% ===== PACKAGES =====
% Essential packages
\usepackage[utf8]{inputenc}           % UTF-8 encoding
\usepackage[T1]{fontenc}              % Font encoding
\usepackage[margin=1in]{geometry}     % Page margins
\usepackage{graphicx}                 % Include images
\usepackage{amsmath,amssymb,amsthm}   % Math support

% Optional but recommended packages
\usepackage{hyperref}                 % Clickable links
\usepackage{cite}                     % Citation management
\usepackage{booktabs}                 % Professional tables
\usepackage{listings}                 % Code listings
\usepackage{xcolor}                   % Colors

% ===== HYPERREF CONFIGURATION =====
\hypersetup{
    colorlinks=true,
    linkcolor=blue,
    citecolor=blue,
    urlcolor=cyan
}

% ===== TITLE PAGE INFORMATION =====
\title{Your Article Title Here}
\author{
    Author Name\\
    \texttt{email@example.com}\\
    Institution Name
}
\date{\today}

% ===== DOCUMENT START =====
\begin{document}

\maketitle

\begin{abstract}
This is the abstract. It should concisely summarize your work, including the problem, method, results, and conclusions. Typically 150-250 words.
\end{abstract}

% ===== SECTIONS =====

\section{Introduction}
\label{sec:introduction}

This is the introduction. Use \texttt{\\label\{sec:name\}} to label sections for cross-referencing. Reference a section like this: Section~\ref{sec:methods}.

Key points to include:
\begin{itemize}
    \item Background and context
    \item Problem statement
    \item Research objectives
    \item Paper organization
\end{itemize}

\section{Methods}
\label{sec:methods}

Describe your methodology here. You can include equations:

\begin{equation}
    E = mc^2
    \label{eq:einstein}
\end{equation}

And reference them: Equation~\ref{eq:einstein} shows Einstein's mass-energy equivalence.

\subsection{Experimental Setup}

Subsections help organize complex sections.

\section{Results}
\label{sec:results}

Present your results here. Include figures and tables:

\begin{figure}[h]
    \centering
    % \includegraphics[width=0.8\textwidth]{figures/example.pdf}
    \caption{Example figure caption. Remove the \% above to include actual image.}
    \label{fig:example}
\end{figure}

Reference figures using: Figure~\ref{fig:example}.

\begin{table}[h]
    \centering
    \caption{Example table with professional formatting}
    \label{tab:results}
    \begin{tabular}{lcc}
        \toprule
        Parameter & Value & Unit \\
        \midrule
        Temperature & 25 & °C \\
        Pressure & 1.0 & atm \\
        Time & 60 & min \\
        \bottomrule
    \end{tabular}
\end{table}

\section{Discussion}
\label{sec:discussion}

Interpret your results and discuss their implications.

\section{Conclusion}
\label{sec:conclusion}

Summarize the main findings and suggest future work.

% ===== BIBLIOGRAPHY =====
% Option 1: Manual bibliography
\begin{thebibliography}{99}
    \bibitem{example}
        Author, A. (2024). \textit{Title of Work}. Publisher.
\end{thebibliography}

% Option 2: BibTeX (uncomment and create references.bib)
% \bibliographystyle{plain}
% \bibliography{references}

\end{document}
